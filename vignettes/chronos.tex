\documentclass[]{article}
\usepackage{lmodern}
\usepackage{amssymb,amsmath}
\usepackage{ifxetex,ifluatex}
\usepackage{fixltx2e} % provides \textsubscript
\ifnum 0\ifxetex 1\fi\ifluatex 1\fi=0 % if pdftex
  \usepackage[T1]{fontenc}
  \usepackage[utf8]{inputenc}
\else % if luatex or xelatex
  \ifxetex
    \usepackage{mathspec}
  \else
    \usepackage{fontspec}
  \fi
  \defaultfontfeatures{Ligatures=TeX,Scale=MatchLowercase}
\fi
% use upquote if available, for straight quotes in verbatim environments
\IfFileExists{upquote.sty}{\usepackage{upquote}}{}
% use microtype if available
\IfFileExists{microtype.sty}{%
\usepackage[]{microtype}
\UseMicrotypeSet[protrusion]{basicmath} % disable protrusion for tt fonts
}{}
\PassOptionsToPackage{hyphens}{url} % url is loaded by hyperref
\usepackage[unicode=true]{hyperref}
\hypersetup{
            pdftitle={Introduction to the chronosphere R package},
            pdfauthor={Adam T. Kocsis, Nussaïbah B. Raja},
            pdfborder={0 0 0},
            breaklinks=true}
\urlstyle{same}  % don't use monospace font for urls
\usepackage[margin=1in]{geometry}
\usepackage{color}
\usepackage{fancyvrb}
\newcommand{\VerbBar}{|}
\newcommand{\VERB}{\Verb[commandchars=\\\{\}]}
\DefineVerbatimEnvironment{Highlighting}{Verbatim}{commandchars=\\\{\}}
% Add ',fontsize=\small' for more characters per line
\usepackage{framed}
\definecolor{shadecolor}{RGB}{248,248,248}
\newenvironment{Shaded}{\begin{snugshade}}{\end{snugshade}}
\newcommand{\KeywordTok}[1]{\textcolor[rgb]{0.13,0.29,0.53}{\textbf{#1}}}
\newcommand{\DataTypeTok}[1]{\textcolor[rgb]{0.13,0.29,0.53}{#1}}
\newcommand{\DecValTok}[1]{\textcolor[rgb]{0.00,0.00,0.81}{#1}}
\newcommand{\BaseNTok}[1]{\textcolor[rgb]{0.00,0.00,0.81}{#1}}
\newcommand{\FloatTok}[1]{\textcolor[rgb]{0.00,0.00,0.81}{#1}}
\newcommand{\ConstantTok}[1]{\textcolor[rgb]{0.00,0.00,0.00}{#1}}
\newcommand{\CharTok}[1]{\textcolor[rgb]{0.31,0.60,0.02}{#1}}
\newcommand{\SpecialCharTok}[1]{\textcolor[rgb]{0.00,0.00,0.00}{#1}}
\newcommand{\StringTok}[1]{\textcolor[rgb]{0.31,0.60,0.02}{#1}}
\newcommand{\VerbatimStringTok}[1]{\textcolor[rgb]{0.31,0.60,0.02}{#1}}
\newcommand{\SpecialStringTok}[1]{\textcolor[rgb]{0.31,0.60,0.02}{#1}}
\newcommand{\ImportTok}[1]{#1}
\newcommand{\CommentTok}[1]{\textcolor[rgb]{0.56,0.35,0.01}{\textit{#1}}}
\newcommand{\DocumentationTok}[1]{\textcolor[rgb]{0.56,0.35,0.01}{\textbf{\textit{#1}}}}
\newcommand{\AnnotationTok}[1]{\textcolor[rgb]{0.56,0.35,0.01}{\textbf{\textit{#1}}}}
\newcommand{\CommentVarTok}[1]{\textcolor[rgb]{0.56,0.35,0.01}{\textbf{\textit{#1}}}}
\newcommand{\OtherTok}[1]{\textcolor[rgb]{0.56,0.35,0.01}{#1}}
\newcommand{\FunctionTok}[1]{\textcolor[rgb]{0.00,0.00,0.00}{#1}}
\newcommand{\VariableTok}[1]{\textcolor[rgb]{0.00,0.00,0.00}{#1}}
\newcommand{\ControlFlowTok}[1]{\textcolor[rgb]{0.13,0.29,0.53}{\textbf{#1}}}
\newcommand{\OperatorTok}[1]{\textcolor[rgb]{0.81,0.36,0.00}{\textbf{#1}}}
\newcommand{\BuiltInTok}[1]{#1}
\newcommand{\ExtensionTok}[1]{#1}
\newcommand{\PreprocessorTok}[1]{\textcolor[rgb]{0.56,0.35,0.01}{\textit{#1}}}
\newcommand{\AttributeTok}[1]{\textcolor[rgb]{0.77,0.63,0.00}{#1}}
\newcommand{\RegionMarkerTok}[1]{#1}
\newcommand{\InformationTok}[1]{\textcolor[rgb]{0.56,0.35,0.01}{\textbf{\textit{#1}}}}
\newcommand{\WarningTok}[1]{\textcolor[rgb]{0.56,0.35,0.01}{\textbf{\textit{#1}}}}
\newcommand{\AlertTok}[1]{\textcolor[rgb]{0.94,0.16,0.16}{#1}}
\newcommand{\ErrorTok}[1]{\textcolor[rgb]{0.64,0.00,0.00}{\textbf{#1}}}
\newcommand{\NormalTok}[1]{#1}
\usepackage{graphicx,grffile}
\makeatletter
\def\maxwidth{\ifdim\Gin@nat@width>\linewidth\linewidth\else\Gin@nat@width\fi}
\def\maxheight{\ifdim\Gin@nat@height>\textheight\textheight\else\Gin@nat@height\fi}
\makeatother
% Scale images if necessary, so that they will not overflow the page
% margins by default, and it is still possible to overwrite the defaults
% using explicit options in \includegraphics[width, height, ...]{}
\setkeys{Gin}{width=\maxwidth,height=\maxheight,keepaspectratio}
\IfFileExists{parskip.sty}{%
\usepackage{parskip}
}{% else
\setlength{\parindent}{0pt}
\setlength{\parskip}{6pt plus 2pt minus 1pt}
}
\setlength{\emergencystretch}{3em}  % prevent overfull lines
\providecommand{\tightlist}{%
  \setlength{\itemsep}{0pt}\setlength{\parskip}{0pt}}
\setcounter{secnumdepth}{0}
% Redefines (sub)paragraphs to behave more like sections
\ifx\paragraph\undefined\else
\let\oldparagraph\paragraph
\renewcommand{\paragraph}[1]{\oldparagraph{#1}\mbox{}}
\fi
\ifx\subparagraph\undefined\else
\let\oldsubparagraph\subparagraph
\renewcommand{\subparagraph}[1]{\oldsubparagraph{#1}\mbox{}}
\fi

% set default figure placement to htbp
\makeatletter
\def\fps@figure{htbp}
\makeatother

\usepackage{etoolbox}
\makeatletter
\providecommand{\subtitle}[1]{% add subtitle to \maketitle
  \apptocmd{\@title}{\par {\large #1 \par}}{}{}
}
\makeatother
% https://github.com/rstudio/rmarkdown/issues/337
\let\rmarkdownfootnote\footnote%
\def\footnote{\protect\rmarkdownfootnote}

% https://github.com/rstudio/rmarkdown/pull/252
\usepackage{titling}
\setlength{\droptitle}{-2em}

\pretitle{\vspace{\droptitle}\centering\huge}
\posttitle{\par}

\preauthor{\centering\large\emph}
\postauthor{\par}

\predate{\centering\large\emph}
\postdate{\par}

\title{Introduction to the `chronosphere' R package}
\author{Adam T. Kocsis, Nussaïbah B. Raja}
\date{2019-11-29}

\begin{document}
\maketitle

\section{1. Introduction}\label{introduction}

\subsection{1.1. Installation}\label{installation}

To install this alpha version of the package, you must download it
either from the CRAN servers or its dedicated GitHub repository
(\url{http://www.github.com/adamkocsis/chronosphere/}). All minor
updates will be posted on GitHub as soon as they are finished, so please
check this regularly. The version on CRAN will be lagging for some time,
as it takes the servers many days to process everything. All questions
and bugs can be reported at the GitHub issues board
(\url{https://github.com/adamkocsis/chronosphere/issues}). Instead of
spending it on actual research, a tremendous amount of time was invested
in making this piece of software streamlined and user-friendly. If you
use a dataset of the package in a publication, please cite both its
reference(s) and the \texttt{chronosphere} package itself.

After installing, from the CRAN, from a source or with
\texttt{devtools::install\_github()} , you can attach the package with:

\begin{Shaded}
\begin{Highlighting}[]
\KeywordTok{library}\NormalTok{(chronosphere)}
\end{Highlighting}
\end{Shaded}

\section{1.2 General features}\label{general-features}

The purpose of the \texttt{chronosphere} project is to facilitate,
streamline and fasten the finding, acquisition and importing of Earth
science data in R. Although the package currently focuses on deep time
data sets, the scope of the included data sets will be much larger,
spanning from a single variable published as supplementary material in a
journal article, to GIS data or the entire output of GCM models.
\texttt{chronosphere} intends to decrease the gap between research
hypotheses and the finding, download and importing of data sets.

\section{2. RasterArray}\label{rasterarray}

Faster data importing and better organization represents a considerable
part of this process. Spatially explicit data are excellent candidates
to demonstrate how more efficient data organization can speed up
research. Although R has an excellent infrastructure for handling raster
data (Hijmans \& van Etten, 2019), the arrangement of individual layers
are rather limited, which can be a problem, when a large number of
layers have to be considered. RasterStacks and RasterBricks are very
efficient for organizing RasterLayers according to a single dimension
(e.g.~depth for 3D variables, or time), multidimensional structures are
preferred.

To offer a more effective solution, the \texttt{chronosphere} package
includes the definition of the RasterArray S4 class. RasterArrays
represent hybrids between RasterStacks and regular R arrays. In short,
they are virtual arrays of RasterLayers, and can be thought of as
regular arrays, that include entire rasters as elements rather than
single numeric, logical or character values. As regular R users are
familiar with subsetting, combinations and structures of regular arrays
(including formal vectors and matrices), the finding, extraction and
storage of spatially explicit data is much easier in such containers.

\subsection{2.1. Structure}\label{structure}

RasterArrays do not directly inherit from Raster* objects of the raster
package, as a considerable number of main functions differ, but they
rather represent a wrapper object around regular RasterStacks - stacks
of individual RasterLayers. This ensures that whenever users are
unfamiliar with the methods of RasterArray class objects, they can
always reduce their data to stacks or individual layers.

The formal class RasterArray has only two slots: a stack and and index.
The stack includes the Raster data in an unfolded manner, similarly to
how matrices and arrays are essentially just vectors with additional
attributes. The stack slot incorporates a single RasterStack object,
which represents the data content of the object. The index slot, on the
other hand, describes the structure of the RasterArray. It is a
vector/matrix/array of integers each representing an index of the layers
in the stack. The configuration (dimensions) of the index represents the
entire array.

The \texttt{chronosphere} package includes two demo data sets: a set of
ancient topographies (Scotese and Wright, 2018) and time series of
bio-climatic variables (annual mean temperature and precipitation) from
the CHELSA project (Karger et al, 2017a, 2017b) These datasets can be
attached with the data command.

\begin{Shaded}
\begin{Highlighting}[]
\KeywordTok{data}\NormalTok{(dems)}
\KeywordTok{data}\NormalTok{(clim)}
\end{Highlighting}
\end{Shaded}

The structure of RasterArrays can be inspected if the object's name is
typed into the console:

\begin{Shaded}
\begin{Highlighting}[]
\NormalTok{dems}
\end{Highlighting}
\end{Shaded}

\begin{verbatim}
## class         : RasterArray 
## RasterLayer properties: 
## - dimensions  : 181, 361  (nrow, ncol)
## - resolution  : 1, 1  (x, y)
## - extent      : -180.5, 180.5, -90.5, 90.5  (xmin, xmax, ymin, ymax)
## - coord. ref. : +proj=longlat +datum=WGS84 +ellps=WGS84 +towgs84=0,0,0 
## Array properties: 
## - dimensions   : 10  (vector)
## - num. layers    : 10
## - proxy:
##         0        5       10       15       20       25       30       35 
##  "dem_0"  "dem_5" "dem_10" "dem_15" "dem_20" "dem_25" "dem_30" "dem_35" 
##       40       45 
## "dem_40" "dem_45"
\end{verbatim}

The first part of the console output includes properties of the
individual RasterLayers stored in the stack. These layers have to share
essential attributes that allow them to be stored in a single stack
(extent, resolution, CRS).

The second part of the output is a visualization of the structure of the
RasterArray itself. In the case of the DEMs, 10 layers are stored in the
stack, each layer having its individual name (e.g. \texttt{dem\_0}). It
is a single dimensional array (vector), and each element has its name in
the array (\texttt{0}). The differentiating between the names of layers
and the names of elements allows different subsetting and replacement
rules for the two, which both can be handy - depending on the needs of
the user.

The structure of the RasterArray can be visualized, analyzed or
processed using the the proxy object. Proxies are essentially the same
as the index slots of the RasterArray, but instead of including the
indices of the layers they represent, proxies include the names of the
layers. These can be accessed using the \texttt{proxy()} function.

\begin{Shaded}
\begin{Highlighting}[]
\KeywordTok{proxy}\NormalTok{(dems)}
\end{Highlighting}
\end{Shaded}

\begin{verbatim}
##        0        5       10       15       20       25       30       35 
##  "dem_0"  "dem_5" "dem_10" "dem_15" "dem_20" "dem_25" "dem_30" "dem_35" 
##       40       45 
## "dem_40" "dem_45"
\end{verbatim}

Proxies are displayed as the second parts of the console output when the
name of the object is typed into the console (show method).

\begin{Shaded}
\begin{Highlighting}[]
\NormalTok{clim}
\end{Highlighting}
\end{Shaded}

\begin{verbatim}
## class         : RasterArray 
## RasterLayer properties: 
## - dimensions  : 90, 180  (nrow, ncol)
## - resolution  : 2, 2  (x, y)
## - extent      : -180, 180, -90, 90  (xmin, xmax, ymin, ymax)
## - coord. ref. : +proj=longlat +datum=WGS84 +ellps=WGS84 +towgs84=0,0,0 
## Array properties: 
## - dimensions  : 10, 2  (nrow, ncol)
## - num. layers    : 20
## - proxy:
##       bio1         bio12       
## 2001 "bio01_2001" "bio12_2001"
## 2002 "bio01_2002" "bio12_2002"
## 2003 "bio01_2003" "bio12_2003"
## 2004 "bio01_2004" "bio12_2004"
## 2005 "bio01_2005" "bio12_2005"
## 2006 "bio01_2006" "bio12_2006"
## 2007 "bio01_2007" "bio12_2007"
## 2008 "bio01_2008" "bio12_2008"
## 2009 "bio01_2009" "bio12_2009"
## 2010 "bio01_2010" "bio12_2010"
\end{verbatim}

This RasterArray has 10 rows (annual means) and two variables/columns:
temperature (\texttt{bio1}) and precipitation (\texttt{bio12}). With the
\texttt{proxy()} function it is easy to interact with this object, or to
query or analyze it.

\begin{Shaded}
\begin{Highlighting}[]
\KeywordTok{proxy}\NormalTok{(clim)}
\end{Highlighting}
\end{Shaded}

\begin{verbatim}
##      bio1         bio12       
## 2001 "bio01_2001" "bio12_2001"
## 2002 "bio01_2002" "bio12_2002"
## 2003 "bio01_2003" "bio12_2003"
## 2004 "bio01_2004" "bio12_2004"
## 2005 "bio01_2005" "bio12_2005"
## 2006 "bio01_2006" "bio12_2006"
## 2007 "bio01_2007" "bio12_2007"
## 2008 "bio01_2008" "bio12_2008"
## 2009 "bio01_2009" "bio12_2009"
## 2010 "bio01_2010" "bio12_2010"
\end{verbatim}

RasterArrays are fairly easy to construct: one only needs a stack of the
data and an regular vector/matrix/array including integers. For
instance, the dems object can be recreated from scratch without any
problem.

\begin{Shaded}
\begin{Highlighting}[]
\CommentTok{# a stack of rasters}
\NormalTok{stackOfLayers <-}\StringTok{ }\NormalTok{dems}\OperatorTok{@}\NormalTok{stack}
\CommentTok{# an index object}
\NormalTok{ind <-}\StringTok{ }\DecValTok{1}\OperatorTok{:}\DecValTok{10}
\KeywordTok{names}\NormalTok{(ind) <-}\StringTok{ }\NormalTok{letters[}\DecValTok{1}\OperatorTok{:}\DecValTok{10}\NormalTok{]}
\CommentTok{# a RasterArray}
\NormalTok{nra  <-}\StringTok{ }\KeywordTok{RasterArray}\NormalTok{(}\DataTypeTok{index=}\NormalTok{ind, }\DataTypeTok{stack=}\NormalTok{stackOfLayers)}
\NormalTok{nra}
\end{Highlighting}
\end{Shaded}

\begin{verbatim}
## class         : RasterArray 
## RasterLayer properties: 
## - dimensions  : 181, 361  (nrow, ncol)
## - resolution  : 1, 1  (x, y)
## - extent      : -180.5, 180.5, -90.5, 90.5  (xmin, xmax, ymin, ymax)
## - coord. ref. : +proj=longlat +datum=WGS84 +ellps=WGS84 +towgs84=0,0,0 
## Array properties: 
## - dimensions   : 10  (vector)
## - num. layers    : 10
## - proxy:
##         a        b        c        d        e        f        g        h 
##  "dem_0"  "dem_5" "dem_10" "dem_15" "dem_20" "dem_25" "dem_30" "dem_35" 
##        i        j 
## "dem_40" "dem_45"
\end{verbatim}

The attributes of the index object are defining the structure of the
RasterArray. RasterArrays can be created with the combination of
individual RasterLayers (or RasterArrays) using the combine() function.

\begin{Shaded}
\begin{Highlighting}[]
\CommentTok{# one raster}
\NormalTok{r1 <-}\StringTok{ }\KeywordTok{raster}\NormalTok{()}
\KeywordTok{values}\NormalTok{(r1) <-}\StringTok{ }\DecValTok{1}
\CommentTok{# same structure, different value}
\NormalTok{r2 <-}\KeywordTok{raster}\NormalTok{()}
\KeywordTok{values}\NormalTok{(r2) <-}\StringTok{ }\DecValTok{2}
\NormalTok{comb <-}\StringTok{ }\KeywordTok{combine}\NormalTok{(r1, r2)}
\NormalTok{comb}
\end{Highlighting}
\end{Shaded}

\begin{verbatim}
## class         : RasterArray 
## RasterLayer properties: 
## - dimensions  : 180, 360  (nrow, ncol)
## - resolution  : 1, 1  (x, y)
## - extent      : -180, 180, -90, 90  (xmin, xmax, ymin, ymax)
## - coord. ref. : +proj=longlat +datum=WGS84 +ellps=WGS84 +towgs84=0,0,0 
## Array properties: 
## - dimensions   : 2  (vector)
## - num. layers    : 2
## - proxy:
##         r1        r2 
## "layer.1" "layer.2"
\end{verbatim}

Matrix-like RasterArrays can also be created easily with the, cbind(),
and rbind() functions.

\begin{Shaded}
\begin{Highlighting}[]
\CommentTok{# bind dems to itself}
\KeywordTok{cbind}\NormalTok{(dems, dems)}
\end{Highlighting}
\end{Shaded}

\begin{verbatim}
## class         : RasterArray 
## RasterLayer properties: 
## - dimensions  : 181, 361  (nrow, ncol)
## - resolution  : 1, 1  (x, y)
## - extent      : -180.5, 180.5, -90.5, 90.5  (xmin, xmax, ymin, ymax)
## - coord. ref. : +proj=longlat +datum=WGS84 +ellps=WGS84 +towgs84=0,0,0 
## Array properties: 
## - dimensions  : 10, 2  (nrow, ncol)
## - num. layers    : 20
## - proxy:
##     [,1]       [,2]      
## 0  "dem_0.1"  "dem_0.2" 
## 5  "dem_5.1"  "dem_5.2" 
## 10 "dem_10.1" "dem_10.2"
## 15 "dem_15.1" "dem_15.2"
## 20 "dem_20.1" "dem_20.2"
## 25 "dem_25.1" "dem_25.2"
## 30 "dem_30.1" "dem_30.2"
## 35 "dem_35.1" "dem_35.2"
## 40 "dem_40.1" "dem_40.2"
## 45 "dem_45.1" "dem_45.2"
\end{verbatim}

\subsection{2.2. RasterArray attributes and function to
query}\label{rasterarray-attributes-and-function-to-query}

Functions that query and change attributes of the RasterArray resemble
general arrays more than Raster* objects. They are connected to the
index slot of the RasterArray and return values accordingly.

The number of elements represented in the RasterArray can be queried
with the \texttt{length()} function:

\begin{Shaded}
\begin{Highlighting}[]
\KeywordTok{length}\NormalTok{(dems)}
\end{Highlighting}
\end{Shaded}

\begin{verbatim}
## [1] 10
\end{verbatim}

This RasterArray has 10 elements. The number of column and row names can
be queried in a similar way:

\begin{Shaded}
\begin{Highlighting}[]
\KeywordTok{nrow}\NormalTok{(clim)}
\end{Highlighting}
\end{Shaded}

\begin{verbatim}
## [1] 10
\end{verbatim}

\begin{Shaded}
\begin{Highlighting}[]
\KeywordTok{ncol}\NormalTok{(clim)}
\end{Highlighting}
\end{Shaded}

\begin{verbatim}
## [1] 2
\end{verbatim}

These functions are summarized in the dim() function. This, however,
unlike the regular \texttt{dim()} method of vectors, return the length
of the RasterArray-vector, rather than just \texttt{NULL}.

\begin{Shaded}
\begin{Highlighting}[]
\KeywordTok{dim}\NormalTok{(dems)}
\end{Highlighting}
\end{Shaded}

\begin{verbatim}
## [1] 10
\end{verbatim}

\begin{Shaded}
\begin{Highlighting}[]
\KeywordTok{dim}\NormalTok{(clim)}
\end{Highlighting}
\end{Shaded}

\begin{verbatim}
## [1] 10  2
\end{verbatim}

The organization of RasterLayers can be greatly facilitated with names.
The \texttt{names()}, \texttt{colnames()}, \texttt{rownames()} and
\texttt{dimnames()} functions work the same way on RasterArrays as if
they were arrays of simple numeric, logical or character values. The
\texttt{names()} function returns the names of individual elements of a
vector-like RasterArray.

\begin{Shaded}
\begin{Highlighting}[]
\KeywordTok{names}\NormalTok{(dems)}
\end{Highlighting}
\end{Shaded}

\begin{verbatim}
##  [1] "0"  "5"  "10" "15" "20" "25" "30" "35" "40" "45"
\end{verbatim}

The \texttt{colnames()} and \texttt{rownames()} functions are more
relevant for matrix-like RasterArrays, such as \texttt{clim}.

\begin{Shaded}
\begin{Highlighting}[]
\KeywordTok{colnames}\NormalTok{(clim)}
\end{Highlighting}
\end{Shaded}

\begin{verbatim}
## [1] "bio1"  "bio12"
\end{verbatim}

\begin{Shaded}
\begin{Highlighting}[]
\KeywordTok{rownames}\NormalTok{(clim)}
\end{Highlighting}
\end{Shaded}

\begin{verbatim}
##  [1] "2001" "2002" "2003" "2004" "2005" "2006" "2007" "2008" "2009" "2010"
\end{verbatim}

All name-related methods can be used for replacement as well. For
instance, you can quickly rename the names of the columns of the
\texttt{clim} object this way:

\begin{Shaded}
\begin{Highlighting}[]
\NormalTok{clim2 <-}\StringTok{ }\NormalTok{clim}
\KeywordTok{colnames}\NormalTok{(clim2) <-}\StringTok{ }\KeywordTok{c}\NormalTok{(}\StringTok{"temp"}\NormalTok{, }\StringTok{"prec"}\NormalTok{)}
\NormalTok{clim2}
\end{Highlighting}
\end{Shaded}

\begin{verbatim}
## class         : RasterArray 
## RasterLayer properties: 
## - dimensions  : 90, 180  (nrow, ncol)
## - resolution  : 2, 2  (x, y)
## - extent      : -180, 180, -90, 90  (xmin, xmax, ymin, ymax)
## - coord. ref. : +proj=longlat +datum=WGS84 +ellps=WGS84 +towgs84=0,0,0 
## Array properties: 
## - dimensions  : 10, 2  (nrow, ncol)
## - num. layers    : 20
## - proxy:
##       temp         prec        
## 2001 "bio01_2001" "bio12_2001"
## 2002 "bio01_2002" "bio12_2002"
## 2003 "bio01_2003" "bio12_2003"
## 2004 "bio01_2004" "bio12_2004"
## 2005 "bio01_2005" "bio12_2005"
## 2006 "bio01_2006" "bio12_2006"
## 2007 "bio01_2007" "bio12_2007"
## 2008 "bio01_2008" "bio12_2008"
## 2009 "bio01_2009" "bio12_2009"
## 2010 "bio01_2010" "bio12_2010"
\end{verbatim}

Just as you would do it with normal arrays, the you can query/rewrite
all names with the \texttt{dimnames()} function, that uses a list to
store the names in every dimension.

\begin{Shaded}
\begin{Highlighting}[]
\KeywordTok{dimnames}\NormalTok{(clim2)[[}\DecValTok{1}\NormalTok{]] <-}\StringTok{ }\DecValTok{1}\OperatorTok{:}\DecValTok{10}
\NormalTok{clim2}
\end{Highlighting}
\end{Shaded}

\begin{verbatim}
## class         : RasterArray 
## RasterLayer properties: 
## - dimensions  : 90, 180  (nrow, ncol)
## - resolution  : 2, 2  (x, y)
## - extent      : -180, 180, -90, 90  (xmin, xmax, ymin, ymax)
## - coord. ref. : +proj=longlat +datum=WGS84 +ellps=WGS84 +towgs84=0,0,0 
## Array properties: 
## - dimensions  : 10, 2  (nrow, ncol)
## - num. layers    : 20
## - proxy:
##     temp         prec        
## 1  "bio01_2001" "bio12_2001"
## 2  "bio01_2002" "bio12_2002"
## 3  "bio01_2003" "bio12_2003"
## 4  "bio01_2004" "bio12_2004"
## 5  "bio01_2005" "bio12_2005"
## 6  "bio01_2006" "bio12_2006"
## 7  "bio01_2007" "bio12_2007"
## 8  "bio01_2008" "bio12_2008"
## 9  "bio01_2009" "bio12_2009"
## 10 "bio01_2010" "bio12_2010"
\end{verbatim}

Besides the names of the elements in the RasterArray, every layer has
its own name in the stack. These can be accessed with \texttt{layers()}
function:

\begin{Shaded}
\begin{Highlighting}[]
\KeywordTok{layers}\NormalTok{(clim)}
\end{Highlighting}
\end{Shaded}

\begin{verbatim}
##  [1] "bio01_2001" "bio01_2002" "bio01_2003" "bio01_2004" "bio01_2005"
##  [6] "bio01_2006" "bio01_2007" "bio01_2008" "bio01_2009" "bio01_2010"
## [11] "bio12_2001" "bio12_2002" "bio12_2003" "bio12_2004" "bio12_2005"
## [16] "bio12_2006" "bio12_2007" "bio12_2008" "bio12_2009" "bio12_2010"
\end{verbatim}

The total number of cells in the RasterLayer or the entire stack can
accessed with the \texttt{ncell()} and \texttt{nvalues()} functions,
respectively.

\begin{Shaded}
\begin{Highlighting}[]
\KeywordTok{ncell}\NormalTok{(dems)}
\end{Highlighting}
\end{Shaded}

\begin{verbatim}
## [1] 65341
\end{verbatim}

\begin{Shaded}
\begin{Highlighting}[]
\KeywordTok{nvalues}\NormalTok{(dems)}
\end{Highlighting}
\end{Shaded}

\begin{verbatim}
## [1] 653410
\end{verbatim}

\subsection{2.3. Subsetting and
replacement}\label{subsetting-and-replacement}

Facilitating the accession of items is the primary purpose of
RasterArrays. These either focus on the layers (stack items, double
bracket operator ``{[}{[}'') or the elements of the RasterArray (single
bracket operator ``{[}'').

\subsubsection{2.3.1 Layer selection - Double bracket
{[}{[}}\label{layer-selection---double-bracket}

This form of subsetting and replacement are inherited from the
RasterStack class. Individual layers can be accessed directly from the
stack using either the position index, the name of the layer or the
logical value pointing to the position. Whichever is used, the
RasterArray wrapper is omitted and output will be a RasterLayer or
RasterStack class object.

A single layer can be accessed using its name, regardless of its
position in the RasterArray. This can be visualized either with the
default \texttt{plot()} or the more general \texttt{mapplot()} function.

\begin{Shaded}
\begin{Highlighting}[]
\NormalTok{one <-}\StringTok{ }\NormalTok{dems[[}\StringTok{"dem_45"}\NormalTok{]]}
\KeywordTok{mapplot}\NormalTok{(one, }\DataTypeTok{col=}\StringTok{"earth"}\NormalTok{)}
\end{Highlighting}
\end{Shaded}

\includegraphics{chronos_files/figure-latex/single-1.pdf}

Returning a single RasterArray. Multiple elements will be the format of
a stack:

\begin{Shaded}
\begin{Highlighting}[]
\NormalTok{dems[[}\KeywordTok{c}\NormalTok{(}\DecValTok{1}\NormalTok{,}\DecValTok{2}\NormalTok{)]]}
\end{Highlighting}
\end{Shaded}

\begin{verbatim}
## class         : RasterArray 
## RasterLayer properties: 
## - dimensions  : 181, 361  (nrow, ncol)
## - resolution  : 1, 1  (x, y)
## - extent      : -180.5, 180.5, -90.5, 90.5  (xmin, xmax, ymin, ymax)
## - coord. ref. : +proj=longlat +datum=WGS84 +ellps=WGS84 +towgs84=0,0,0 
## Array properties: 
## - dimensions   : 2  (vector)
## - num. layers    : 2
## - proxy:
##  [1] "dem_0" "dem_5"
\end{verbatim}

Which are the first two RasterLayers in the stack of the RasterArray.

Double brackets can also be used for replacements, but as this has no
effect on the structure of the array, changes implemented with this
method are more difficult to trace. For instance,

\begin{Shaded}
\begin{Highlighting}[]
\CommentTok{# copy}
\NormalTok{dem2 <-}\StringTok{ }\NormalTok{dems}
\NormalTok{dem2[[}\StringTok{"dem_0"}\NormalTok{]] <-}\StringTok{ }\NormalTok{dem2[[}\StringTok{"dem_5"}\NormalTok{]]}
\end{Highlighting}
\end{Shaded}

will rewrite the values in the first element of \texttt{dem2}, but that
will not be evident in the RasterArray's structure.

\begin{Shaded}
\begin{Highlighting}[]
\CommentTok{# but these two are now the same}
\NormalTok{dem2[[}\DecValTok{1}\NormalTok{]]}
\end{Highlighting}
\end{Shaded}

\begin{verbatim}
## class      : RasterLayer 
## dimensions : 181, 361, 65341  (nrow, ncol, ncell)
## resolution : 1, 1  (x, y)
## extent     : -180.5, 180.5, -90.5, 90.5  (xmin, xmax, ymin, ymax)
## crs        : +proj=longlat +datum=WGS84 +ellps=WGS84 +towgs84=0,0,0 
## source     : memory
## names      : dem_0 
## values     : -7000, 6300  (min, max)
## zvar       : z
\end{verbatim}

\begin{Shaded}
\begin{Highlighting}[]
\NormalTok{dem2[[}\DecValTok{2}\NormalTok{]]}
\end{Highlighting}
\end{Shaded}

\begin{verbatim}
## class      : RasterLayer 
## dimensions : 181, 361, 65341  (nrow, ncol, ncell)
## resolution : 1, 1  (x, y)
## extent     : -180.5, 180.5, -90.5, 90.5  (xmin, xmax, ymin, ymax)
## crs        : +proj=longlat +datum=WGS84 +ellps=WGS84 +towgs84=0,0,0 
## source     : memory
## names      : dem_5 
## values     : -7000, 6300  (min, max)
## zvar       : z
\end{verbatim}

\subsubsection{2.3.2 Single bracket}\label{single-bracket}

Features offered by the double bracket (``{[}{[}'') operator are
virtually identical with those of RasterStacks. The true utility of
RasterArrays become evident with simple array-type subsetting.

Unlike Raster* objects of the raster package, single brackets will get
and replace items from the RasterArray as if they were simple arrays.
For example, single elements of the DEMs can be selected with the age of
the DEM, passed as a character subscript.

\begin{Shaded}
\begin{Highlighting}[]
\NormalTok{dems[}\StringTok{"30"}\NormalTok{]}
\end{Highlighting}
\end{Shaded}

\begin{verbatim}
## class      : RasterLayer 
## dimensions : 181, 361, 65341  (nrow, ncol, ncell)
## resolution : 1, 1  (x, y)
## extent     : -180.5, 180.5, -90.5, 90.5  (xmin, xmax, ymin, ymax)
## crs        : +proj=longlat +datum=WGS84 +ellps=WGS84 +towgs84=0,0,0 
## source     : memory
## names      : dem_30 
## values     : -8000, 10200  (min, max)
## zvar       : z
\end{verbatim}

returning the 30Ma RasterLayer. By default, the RasterArray container is
dropped, but it can be conserved, if the drop argument is set to
\texttt{FALSE}.

\begin{Shaded}
\begin{Highlighting}[]
\NormalTok{dem30 <-}\StringTok{ }\NormalTok{dems[}\StringTok{"30"}\NormalTok{, drop=}\OtherTok{FALSE}\NormalTok{]}
\KeywordTok{class}\NormalTok{(dem30)}
\end{Highlighting}
\end{Shaded}

\begin{verbatim}
## [1] "RasterArray"
## attr(,"package")
## [1] "chronosphere"
\end{verbatim}

Beyond bounds accessing is valid for single dimensional RasterArrays
(vector-like ones):

\begin{Shaded}
\begin{Highlighting}[]
\NormalTok{dems[}\DecValTok{4}\OperatorTok{:}\DecValTok{12}\NormalTok{]}
\end{Highlighting}
\end{Shaded}

\begin{verbatim}
## class         : RasterArray 
## RasterLayer properties: 
## - dimensions  : 181, 361  (nrow, ncol)
## - resolution  : 1, 1  (x, y)
## - extent      : -180.5, 180.5, -90.5, 90.5  (xmin, xmax, ymin, ymax)
## - coord. ref. : +proj=longlat +datum=WGS84 +ellps=WGS84 +towgs84=0,0,0 
## Array properties: 
## - dimensions   : 9  (vector)
## - num. layers    : 7
## - proxy:
##        15       20       25       30       35       40       45     <NA> 
## "dem_15" "dem_20" "dem_25" "dem_30" "dem_35" "dem_40" "dem_45"       NA 
##     <NA> 
##       NA
\end{verbatim}

Missing values are legitimate parts of RasterArrays. These gaps in the
data are not represented in the stacks, but only in the index slots of
the RasterArrays. They can be inserted or added into the layers.

\begin{Shaded}
\begin{Highlighting}[]
\NormalTok{demna <-}\StringTok{ }\NormalTok{dems}
\NormalTok{demna[}\DecValTok{3}\NormalTok{] <-}\StringTok{ }\OtherTok{NA}
\end{Highlighting}
\end{Shaded}

Multidimensional subscripts work in a similar fashion. If a single layer
is desired from the RasterArray, that can be accessed using the names of
the margins.

\begin{Shaded}
\begin{Highlighting}[]
\CommentTok{# character is necessary, as the row named "2003" is necessary}
\NormalTok{one <-}\StringTok{ }\NormalTok{clim[}\StringTok{"2003"}\NormalTok{, }\StringTok{"bio1"}\NormalTok{]}
\KeywordTok{mapplot}\NormalTok{(one)}
\end{Highlighting}
\end{Shaded}

\includegraphics{chronos_files/figure-latex/cellsbu-1.pdf}

similarly to entire rows,

\begin{Shaded}
\begin{Highlighting}[]
\NormalTok{clim[}\StringTok{"2005"}\NormalTok{, ]}
\end{Highlighting}
\end{Shaded}

\begin{verbatim}
## class         : RasterArray 
## RasterLayer properties: 
## - dimensions  : 90, 180  (nrow, ncol)
## - resolution  : 2, 2  (x, y)
## - extent      : -180, 180, -90, 90  (xmin, xmax, ymin, ymax)
## - coord. ref. : +proj=longlat +datum=WGS84 +ellps=WGS84 +towgs84=0,0,0 
## Array properties: 
## - dimensions   : 2  (vector)
## - num. layers    : 2
## - proxy:
##          bio1        bio12 
## "bio01_2005" "bio12_2005"
\end{verbatim}

or columns

\begin{Shaded}
\begin{Highlighting}[]
\NormalTok{clim[,}\StringTok{"bio12"}\NormalTok{]}
\end{Highlighting}
\end{Shaded}

\begin{verbatim}
## class         : RasterArray 
## RasterLayer properties: 
## - dimensions  : 90, 180  (nrow, ncol)
## - resolution  : 2, 2  (x, y)
## - extent      : -180, 180, -90, 90  (xmin, xmax, ymin, ymax)
## - coord. ref. : +proj=longlat +datum=WGS84 +ellps=WGS84 +towgs84=0,0,0 
## Array properties: 
## - dimensions   : 10  (vector)
## - num. layers    : 10
## - proxy:
##          2001         2002         2003         2004         2005         2006 
## "bio12_2001" "bio12_2002" "bio12_2003" "bio12_2004" "bio12_2005" "bio12_2006" 
##         2007         2008         2009         2010 
## "bio12_2007" "bio12_2008" "bio12_2009" "bio12_2010"
\end{verbatim}

\subsubsection{2.4 Inherited from Raster*}\label{inherited-from-raster}

As the spatial information is contained entirely in the RasterStacks, a
number methods are practically inherited from RasterStack class. For
instance, all RasterLayer of the RasterArray can be cropped in a single
line of code.

\begin{Shaded}
\begin{Highlighting}[]
\CommentTok{# crop to Australia}
\NormalTok{ext <-}\StringTok{ }\KeywordTok{extent}\NormalTok{(}\KeywordTok{c}\NormalTok{(                }
  \DataTypeTok{xmin =} \FloatTok{106.58}\NormalTok{,}
  \DataTypeTok{xmax =} \FloatTok{157.82}\NormalTok{,}
  \DataTypeTok{ymin =} \FloatTok{-45.23}\NormalTok{,}
  \DataTypeTok{ymax =} \FloatTok{1.14} 
\NormalTok{)) }

\CommentTok{# cropping all DEMS (Australia drifted in)}
\NormalTok{au<-}\StringTok{ }\KeywordTok{crop}\NormalTok{(dems, ext)}

\CommentTok{# select the first element}
\KeywordTok{mapplot}\NormalTok{(au[}\DecValTok{1}\NormalTok{], }\DataTypeTok{col=}\StringTok{"earth"}\NormalTok{)}
\end{Highlighting}
\end{Shaded}

\includegraphics{chronos_files/figure-latex/crop-1.pdf}

Other functions such as aggregation or resampling works just the same.

\begin{Shaded}
\begin{Highlighting}[]
\NormalTok{template <-}\StringTok{ }\KeywordTok{raster}\NormalTok{(}\DataTypeTok{res=}\DecValTok{5}\NormalTok{)}

\CommentTok{# resample all DEMS}
\NormalTok{coarse <-}\StringTok{ }\KeywordTok{resample}\NormalTok{(dems, template)}

\CommentTok{# plot an elemnt}
\KeywordTok{mapplot}\NormalTok{(coarse[}\StringTok{"45"}\NormalTok{], }\DataTypeTok{col=}\StringTok{"earth"}\NormalTok{)}
\end{Highlighting}
\end{Shaded}

\includegraphics{chronos_files/figure-latex/resam-1.pdf}

\section{3. Plotting}\label{plotting}

The \texttt{mapplot()} function makes it easy to visually pleasing plots
of a Raster* object.

\subsection{3.1 Colour palettes}\label{colour-palettes}

The package include several colour palettes which can be used for
plotting purposes. An additional palette option developed specifically
with DEMs in mind is \emph{earth}. This combines the \emph{ocean} and
\emph{terra} palettes and automatically sets the breaks to differentiate
between marine and terrestrial cells. An example is show below in
\protect\hyperlink{RasterLayer}{RasterLayer plotting example}.

\begin{Shaded}
\begin{Highlighting}[]
\KeywordTok{showPal}\NormalTok{()}
\end{Highlighting}
\end{Shaded}

\includegraphics{chronos_files/figure-latex/palette-1.pdf}

\hypertarget{RasterLayer}{\subsection{3.2
RasterLayer}\label{RasterLayer}}

The \texttt{mapplot()} function for RasterLayer works similar to the
\texttt{plot()} function. The \texttt{mapplot} function for the Raster*
objects include a default palette, omits the legend, axes and the
bounding box.

\begin{Shaded}
\begin{Highlighting}[]
\KeywordTok{data}\NormalTok{(dems)}
\KeywordTok{mapplot}\NormalTok{(dems[}\DecValTok{1}\NormalTok{])}
\end{Highlighting}
\end{Shaded}

\begin{center}\includegraphics[width=0.8\linewidth]{chronos_files/figure-latex/rlayer_plot-1} \end{center}

\begin{Shaded}
\begin{Highlighting}[]
\CommentTok{#using an custom colour palette}
\KeywordTok{mapplot}\NormalTok{(dems[}\DecValTok{1}\NormalTok{], }\DataTypeTok{col=}\StringTok{"earth"}\NormalTok{, }\DataTypeTok{main=}\StringTok{"Using the earth palette"}\NormalTok{)}
\end{Highlighting}
\end{Shaded}

\begin{center}\includegraphics[width=0.8\linewidth]{chronos_files/figure-latex/rlayer_plot-2} \end{center}

\subsection{3.3 RasterArray}\label{rasterarray-1}

RasterArrays can be plotted as a multi-faceted plots using
\texttt{mapplot()}. The \texttt{mapplot()} function keeps the structure
of the RasterArray in terms of the order that the plots are generate.
The plot titles are automatically generated based on the names of the
layers within the RasterArray. This can be changed by passing the custom
plot titles to the argument \texttt{plot.title}. \textbf{Note:} This
number of plots titles provided should be equal to the number of layers
in the RasterArray.

\begin{Shaded}
\begin{Highlighting}[]
\KeywordTok{data}\NormalTok{(dems)}
\KeywordTok{mapplot}\NormalTok{(dems)}
\end{Highlighting}
\end{Shaded}

\begin{center}\includegraphics[width=0.8\linewidth]{chronos_files/figure-latex/rarray_plot_def-1} \end{center}

\subsubsection{3.3.1 Adding a single consistent
legend}\label{adding-a-single-consistent-legend}

Just like with \protect\hyperlink{RasterLayer}{RasterLayers}, different
palettes can be used for the plots. This implements a single palette for
all the plots. In addition, this can then be used to generate a single
legend that ensures consistency and makes it easier to compare the
figures.

\begin{Shaded}
\begin{Highlighting}[]
\KeywordTok{data}\NormalTok{(dems)}
\KeywordTok{mapplot}\NormalTok{(dems, }\DataTypeTok{col=}\StringTok{"ocean"}\NormalTok{, }\DataTypeTok{legend=}\OtherTok{TRUE}\NormalTok{, }\DataTypeTok{legend.title=}\StringTok{"DEM"}\NormalTok{, }
        \DataTypeTok{plot.title=}\KeywordTok{proxy}\NormalTok{(dems))}
\end{Highlighting}
\end{Shaded}

\begin{center}\includegraphics{chronos_files/figure-latex/rarray_plot_leg-1} \end{center}

\subsubsection{3.3.2 RasterArrays with NA
layers}\label{rasterarrays-with-na-layers}

In the event that one of the layers of the dem does not exist i.e.~has
an NA value instead, this plot corresponding to the this NA layer will
show ``Plot Not Available''.

\begin{Shaded}
\begin{Highlighting}[]
\KeywordTok{data}\NormalTok{(dems)}

\CommentTok{#create NA layer}
\NormalTok{dems[}\DecValTok{5}\NormalTok{] <-}\StringTok{ }\OtherTok{NA}
\KeywordTok{is.na}\NormalTok{(dems)}
\end{Highlighting}
\end{Shaded}

\begin{verbatim}
##     0     5    10    15    20    25    30    35    40    45 
## FALSE FALSE FALSE FALSE  TRUE FALSE FALSE FALSE FALSE FALSE
\end{verbatim}

\begin{Shaded}
\begin{Highlighting}[]
\KeywordTok{mapplot}\NormalTok{(dems, }\DataTypeTok{col=}\StringTok{"coldhot"}\NormalTok{)}
\end{Highlighting}
\end{Shaded}

\begin{center}\includegraphics{chronos_files/figure-latex/na_layer-1} \end{center}

\subsubsection{3.3.3 Number of columns and
pages}\label{number-of-columns-and-pages}

The custom number of columns can also be specified through the
\texttt{ncol} argument. The default number of columns is 3.

\begin{Shaded}
\begin{Highlighting}[]
\CommentTok{# 4 columns}
\KeywordTok{data}\NormalTok{(dems)}
\KeywordTok{mapplot}\NormalTok{(dems, }\DataTypeTok{ncol=}\DecValTok{4}\NormalTok{)}
\end{Highlighting}
\end{Shaded}

\includegraphics{chronos_files/figure-latex/rarray_plot_ncol-1.pdf}

Sometimes, there might be too many plots for them to all plot on one
single page. In this instance, the argument \texttt{multi} can be set to
\texttt{TRUE} to allow the figures to be plotted on multiple figures.

\begin{Shaded}
\begin{Highlighting}[]
\KeywordTok{data}\NormalTok{(dems)}
\KeywordTok{mapplot}\NormalTok{(dems, }\DataTypeTok{multi=}\OtherTok{TRUE}\NormalTok{)}
\end{Highlighting}
\end{Shaded}

\subsubsection{3.3.4 Multiple variables in
RasterArray}\label{multiple-variables-in-rasterarray}

The above described methods can also be applied to RasterArrays
containing more than one variables. However, in this instance, the
number of columns automatically defaults to the number of variables in
the array. Also, plot titles for each individual plot is not printed.
Instead, row labels corresponding to the row names of each variable is
added and when \texttt{legend=TRUE} the default legend title defaults to
the variables names in the RasterArray.

\begin{Shaded}
\begin{Highlighting}[]
\KeywordTok{data}\NormalTok{(clim)}

\KeywordTok{mapplot}\NormalTok{(clim[}\DecValTok{1}\OperatorTok{:}\DecValTok{2}\NormalTok{,], }\DataTypeTok{legend=}\OtherTok{TRUE}\NormalTok{)}
\end{Highlighting}
\end{Shaded}

\includegraphics{chronos_files/figure-latex/rarray_plot_nvar-1.pdf}

The row labels and legend title can be edited by using the argument
\texttt{rowlabels} and \texttt{legend.title} respectively. Different
colours for each variables can also be provided. The number of provided
colour palettes should be either 1 or correspond to the number of
variables.

\begin{Shaded}
\begin{Highlighting}[]
\KeywordTok{data}\NormalTok{(clim)}

\KeywordTok{mapplot}\NormalTok{(clim[}\DecValTok{1}\OperatorTok{:}\DecValTok{2}\NormalTok{,], }\DataTypeTok{col=}\KeywordTok{c}\NormalTok{(}\StringTok{"coldhot"}\NormalTok{, }\StringTok{"wet"}\NormalTok{), }
        \DataTypeTok{legend=}\OtherTok{TRUE}\NormalTok{, }\DataTypeTok{legend.title=}\KeywordTok{c}\NormalTok{(}\StringTok{"Temperature"}\NormalTok{, }\StringTok{"Precipitation"}\NormalTok{))}
\end{Highlighting}
\end{Shaded}

\includegraphics{chronos_files/figure-latex/rarray_plot_rowlabs-1.pdf}

\section{References}\label{references}

Hijmans, R. J., \& van Etten, J. (2019). raster: Geographic Data
Analysis and Modeling. Retrieved from
\url{https://cran.r-project.org/package=raster}

Karger, D. N., Conrad, O., Böhner, J., Kawohl, T., Kreft, H.,
Soria-Auza, R. W., \ldots{} Kessler, M. (2017a). Data from:
Climatologies at high resolution for the earth's land surface areas.
Dryad Digital Repository. \url{https://doi.org/10.5061/dryad.kd1d4}

Karger, D. N., Conrad, O., Böhner, J., Kawohl, T., Kreft, H.,
Soria-Auza, R. W., \ldots{} Kessler, M. (2017b). Climatologies at high
resolution for the earth's land surface areas. Scientific Data, 4(1),
170122. \url{https://doi.org/10.1038/sdata.2017.122}

Scotese, C. R. Wright, N. (2018). PALEOMAP Paleodigital Elevation Models
(PaleoDEMS) for the Phanerozoic. URL:
\url{https://www.earthbyte.org/paleodem-resource-scotese-and-wright-2018/}

\end{document}
